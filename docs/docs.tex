\documentclass[a4paper,12pt,oneside]{article}
\usepackage[slovak]{babel}
\usepackage[left=1.5cm,text={18cm, 25cm},top=2.5cm]{geometry}
\usepackage[IL2]{fontenc}
\usepackage[utf8]{inputenc}
\usepackage{svg, amsmath, amsthm, amssymb}
\usepackage{times}

\theoremstyle{definition}
\newtheorem{definice}{Definice}

\theoremstyle{definition}
\newtheorem{veta}{Věta}

\theoremstyle{definition}
\newtheorem{dukaz}{Důkaz}

\begin{document}
\begin{titlepage}
  \begin{center}
    \includegraphics[width=180mm]{assets/logo}\\[50mm]
    {\Huge Dokumentácia k projektu ISA}\\[5mm]
    {\Huge HTTP nástenka}\\
    \vspace{\stretch{0.618}}
    \begin{flushleft}
      \Large{Matej Soroka (xsorok02) \\ 2019/20}
    \end{flushleft}
  \end{center}
\end{titlepage}

\tableofcontents
\newpage

\section{Návrh aplikácie}

\subsection{Client}
Obsahuje spracovanie argumentov, ktoré reprezentujú IP adresu, port serveru a príslušný príkaz na obsluhu nástenky. Tvorí štart programu kde vytvorí spojenie medzi serverom, vytvorí objekt Request a zašle ho ako požiadavok na server v tvare HTTP requestu.

\subsection{Server}
Program, ktorý zachytáva požiadavky v cykle, takže kým sa požiadavka nevykoná, je príjmanie nových požiadaviek pozastavené, a čaká sa na vykonanie. Keďže vykonávanie operácie trvá okamih, nie je nutné tento problém riesiť do hĺbky. V prípade veľkého množstva požiadaviek je jedno z riešením použitie paralelizácie a samotné úpravy násteniek vykonávať paralelne, no tiaktiež je nutné zaistiť synchronizáciu procesov pri úprave tabuľky.  \\

Server odchytí objekt Request v tvare HTTP requestu a pošle ho objektu Handler, ktorý podľa parametrov požiadavky vykoná operácie nad objektom Dashboard a vygeneruje odpoveď vo forme objektu Response podľa toho ako skončí daná operácia nad objektom Dashboard. \\

Keďže na nástenke nepredpokladáme dlhé správy, tak limit pre dĺžku HTTP požiadavky je 8192 znakov, týmto limitom zabezpečíme či už rýchlosť vykonania požiadavky, tak aj šetrnosť pamäte.

\subsection{Request}
Trieda pre ucelenie požiadavky na server, vytvára sa na strane klienta a zasiela sa server.

\subsection{Response}
Trieda pre ucelenie odpovede pre sieťového klienta, vytvára sa na strane serveru.

\subsection{Handler}
Trieda obsahujúca spracovanie požiadavky a podľa nej určí, aká operácia sa má nad objektom Dashoard vykonať, samotný objekt Handler volá metódy objektu Dashboard s príslušnými parametrami.

\subsection{Dashboard}
Trieda obsahuje samotnú logiku nástenky, ktorá obsahuje tabule a príspevky a vyhodnocuje danú požiadavku na základe ktorej nastavuje parametre odpovede pre klienta.

\subsection{Post}
Trieda slúži ako ucelenie príspevku na nástenke.

\end{document}
